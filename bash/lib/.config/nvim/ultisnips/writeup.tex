\title{CS 383 - Machine Learning}
\author{
        Assignment 1 - Dimensionality Reduction
}
\date{}
\documentclass[12pt]{article}
\usepackage[margin=0.7in]{geometry}
\usepackage{graphicx}
\usepackage{float}
\usepackage{amsmath}

\begin{document}
\maketitle

\section*{Introduction}
In this assignment, in addition to related theory/math questions, you'll work on visualizing data and reducing its dimensionality.\\

\noindent
You may not use any functions from machine learning library in your code, however you may use statistical functions.  For example, if available you \textbf{MAY NOT} use functions like
\begin{itemize}
\item pca
\item entropy
\end{itemize}

\noindent
however you \textbf{MAY} use basic statistical functions like:
\begin{itemize}
\item std
\item mean
\item cov
\item svd
\end{itemize}


\section*{Grading}
Although all assignments will be weighed equally in computing your homework grade, below is the grading rubric we will use for this assignment:

\begin{table}[h]
\begin{centering}
\begin{tabular}{|l|l|}
\hline
Part 1 (Theory) & 15pts \\
Part 2 (PCA) & 40pts\\
Part 3 (Eigenfaces) & 30pts\\
Report & 15pts\\
\hline
\textbf{TOTAL} & 100pts\\
\hline
\end{tabular}
\caption{Grading Rubric}
\end{centering}
\end{table}

\newpage
\section*{DataSets}

\paragraph{Yale Faces Datasaet}
This dataset consists of 154 images (each of which is 243x320 pixels) taken from 14 people at 11 different viewing conditions (for our purposes, the first person was removed from the official dataset so person ID=2 is the first person).\\

\noindent
The filename of each images encode class information:
\begin{center}
subject$<ID>$.$<condition>$
\end{center}
Data obtained from:  http://cvc.cs.yale.edu/cvc/projects/yalefaces/yalefaces.html

\newpage
\section{Theory Questions}
\begin{enumerate}


\item (15 points) Consider the following data:\\
\begin{center}
Class 1 =
$
 \begin{bmatrix}
	-2 & 1\\
	-5 & -4\\
	-3 & 1\\
	0 & 3\\
	-8 & 11\\

\end{bmatrix}
$
, Class 2 =
$
 \begin{bmatrix}
	-2 & 5\\
	1 & 0\\
	5 & -1\\
	-1 & -3\\
	6 & 1\\
\end{bmatrix}
$
\end{center}
	\begin{enumerate}
	\item Compute the information gain for each feature.  You could standardize the data overall, although it won't make a difference. (13pts).
	\item Which feature is more discriminating based on results in Part (a) (2pt)?
	\end{enumerate}

\end{enumerate}


\newpage
\section{(40pts) Dimensionality Reduction via PCA}\label{pca}
Download and extract the dataset \emph{yalefaces.zip} from Blackboard.  This dataset has 154 images ($N=154$) each of which is a 243x320 image ($D=77760$).  In order to process this data your script will need to:

\begin{enumerate}
\item Read in the list of files
\item Create a 154x1600 data matrix such that for each image file
	\begin{enumerate}
	\item Read in the image as a 2D array (234x320 pixels)
	\item Subsample the image to become a 40x40 pixel image (for processing speed).  I suggest you use your image processing library to do this for you.
	\item \emph{Flatten} the image to a 1D array (1x1600)
	\item Concatenate this as a row of your data matrix.
	\end{enumerate}
\end{enumerate}

\noindent
Once you have your data matrix, your script should:
\begin{enumerate}
  \item Standardizes the data
  \item Reduces the data to 2D using PCA
  \item Graphs the data for visualization
\end{enumerate}

\noindent
Recall that although you may not use any package ML functions like \emph{pca}, you \textbf{may} use statistical functions like \emph{svd}.\\

\noindent
Your graph should end up looking similar to Figure \ref{PCA} (although it may be rotated differently, depending how you ordered things).
\begin{figure}[H]
\begin{center}
\includegraphics{Part1_faces.png}
\caption{2D PCA Projection of data}
\label{PCA}
\end{center}
\end{figure}

\textbf{NOTE:} Depending on your linear algebra package, the eigenvectors may have the opposite direction.  This is fine since technically an eigenvector multiplied by any scalar are equivalent.

\newpage
\section{(30 points) Eigenfaces}\label{eigenface}
Download and extract the dataset \emph{yalefaces.zip} from Blackboard.  This dataset has 154 images ($N=154$) each of which is a 243x320 image ($D=77760$).  In order to process this data your script will need to:

\begin{enumerate}
\item Read in the list of files
\item Create a 154x1600 data matrix such that for each image file
	\begin{enumerate}
	\item Read in the image as a 2D array (234x320 pixels)
	\item Subsample the image to become a 40x40 pixel image (for processing speed)
	\item \emph{Flatten} the image to a 1D array (1x1600)
	\item Concatenate this as a row of your data matrix.
	\end{enumerate}
\end{enumerate}

\noindent
\paragraph{Write a script that:}
\begin{enumerate}
  \item Imports the data as mentioned above.
  \item Standardizes the data.
  \item Performs PCA on the data (again, although you may not use any package ML functions like \emph{pca}, you \textbf{may} use statistical functions like \emph{svd}).
    \item Visualizes the most important principle component as a 40x40 image (see Figure \ref{eigenface1}).
    \item Reconstructs \emph{subject02.centerlight} using the primary principle component.  To best see the full re-construction, ``unstandardize" the reconstruction by multiplying it by the original standard deviation and adding back in the original mean.
  \item Determines the number of principle components necessary to encode at least 95\% of the information, $k$.
 \item Uses the $k$ most significant eigen-vectors reconstructs \emph{subject02.centerlight}  (see Figure \ref{recon2}).  For the fun of it maybe even look to see if you can perfectly reconstruct the face if you use all the eigen-vectors!  Again, to best see the full re-construction, ``unstandardize" the reconstruction by multiplying it by the original standard deviation and adding back in the original mean.
\end{enumerate}

\noindent
\textbf{NOTE}:  In order to view your matrix as an image you may need to adjust the values to fit in the range that your image viewing function expects.  Read its documentation.\\

\noindent
Your principle eigenface should end up looking similar to Figure \ref{eigenface1}.
\begin{figure}[H]
\begin{center}
\includegraphics{eigenface1.png}
\caption{Primary Principle Component}
\label{eigenface1}
\end{center}
\end{figure}


Your reconstruction should end up looking similar to Figure \ref{recon2}

\begin{figure}[H]
\begin{center}
\includegraphics[width=\textwidth]{reconstruction_unstandardized.png}
\caption{Reconstruction of first person, post-unstandardized (ID=2)}
\label{recon2}
\end{center}
\end{figure}

\newpage
\section*{Submission}
For your submission, upload to Blackboard a single zip file containing:

\begin{enumerate}
\item PDF Writeup
\item Source Code
\item readme.txt file
\item You do not need to include the images.
\item You \textbf{no not} need to include the dataset.  HOWEVER, it should be clear in your script (and/or readme) where your code expects the dataset to reside.
\end{enumerate}

\noindent
The readme.txt file should contain information on how to run your code to reproduce results for each part of the assignment.  In particular for this assignment, it should also indicate where the \emph{yalefaces} directory should be in order to run.  Do not include spaces or special characters (other than the underscore character) in your file and directory names.  Doing so may break our grading scripts.\\

\noindent
The PDF document should contain the following:
\begin{enumerate}
\item Part 1: Your answers to the theory questions.
\item Part 2: The visualization of the PCA result
\item Part 3:
	\begin{enumerate}
	\item Number of principle components needed to represent 95\% of information, $k$.
	\item Visualization of primary principle component
	\item Visualization of the reconstruction of the first person using
		\begin{enumerate}
		\item Original image
		\item Single principle component
		\item $k$ principle components.
		\end{enumerate}
	\end{enumerate}
\end{enumerate}

\end{document}

